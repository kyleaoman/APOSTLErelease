
%|-------------------------------------------------------|
%| Description of database server and technical aspects. |
%|-------------------------------------------------------|
\subsection{Technical aspects and infrastructure}
\label{subsection:technical_aspects}

Multiple layouts and frameworks are available for storing large datasets (such
as \textsc{MongoDB}\footnote{\url{https://www.mongodb.org/}},
\textsc{SciDB}\footnote{\url{http://www.paradigm4.com/}},
\textsc{Hadoop}\footnote{\url{http://hadoop.apache.org/}}, ...) each coming with
advantages and shortcomings. In the case of galaxy catalogues extracted from
cosmological simulations, the \textit{Millennium simulation} used an \sql
database for its public release and the wider astronomy community has since
developed a familiarity with its structure and way to query the data. To allow
users the simplest transition between databases, we have adopted the same
framework and a similar table design as the \textit{Millennium simulation} \sql
database \citep{Lemson2006b}. More efficient ways of querying the data could
exist, with differing database formats or table structures, however we decided
that maintaining the familiar aspects of previous databases outweighs the
potential performance gains.

The server hosting the front end web interface operates on Centos linux, running
\textsc{Apache Tomcat 6.0.24}. This server interfaces with the database host,
submitting queries and having their results streamed via a Java web application
(originally written for the \textit{Millennium simulation}). The database itself
is stored on a single physical Windows Server 2008 system with 128~GB of ram,
80~TB of disk storage and two Xeon E5-2670 CPUs which runs Microsoft SQL Server
2012. The main table for the largest simulation contains 65,996,151 rows, which
corresponds to $\approx300~\rm{GBytes}$ of disk space.

Columns are indexed on disk as follows (see below for the description of the
content of each table):

\begin{enumerate}

\item The \textbf{SubHalo} and \textbf{Sizes} tables have a clustered index on
  the \GalaxyID. This allows joins between the tables and queries for
  progenitors and descendants to run efficiently. \GalaxyID rows are assigned
  such that progenitors of each galaxy have a continuous range of \GalaxyID.

\item The \textbf{SubHalo} tables have an additional index on (\texttt{Snapnum},
  \GalaxyID) due to the common nature of queries that request a particular time
  in the simulation.

\item The \textbf{Aperture} tables have a clustered index on the combination of
  (\GalaxyID, \texttt{ApertureSize}) and \\(\texttt{ApertureSize}, \GalaxyID) to
  aid queries searching for all information about a single galaxy or one
  aperture size for many galaxies respectively.

\item The \textbf{FOF} tables are clustered on the combination
  (\texttt{SnapNum}, \texttt{GroupID}), which uniquely identifies the FoF group
  and can be used to join to the \textbf{SubHalo} table.

\end{enumerate}

Typical queries (such as the ones given as examples in
Sec.~\ref{database_usage}) take a few milliseconds to complete on the
server.  More complex queries (i.e. joining multiple tables or
navigating the merger trees for multiple galaxies at the same time)
can take up to a few seconds. As the usage goes up, additional
indexing of the columns could be added to improve the performance of
common, more complex, queries.

The mock \emph{gri} images have been processed once for the entire simulation
and are stored on a separate server. When querying images, the \sql server
generates valid HTML tags containing the links to the images. No caching has
been put in place but such facility could easily be added in case of large
demand.





