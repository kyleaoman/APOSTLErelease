%|-------------|
%| Conclusions |
%|-------------|
\section{Conclusions}
\label{conclusion}
This paper introduces a public \sql relational database\footnote{Available at
  the address \dbAddress} containing the integrated quantities and merger
histories for more than $10^5$ galaxies from the \eagle suite of hydrodynamic
simulations. The database contains all the galaxies from the largest \eagle
simulation as well as galaxies from smaller volumes where the resolution and AGN
model were varied. The details of these simulations are presented by
\citet{Schaye2015} and a list of published results using the simulation can be
found on our websites\footnote{\url{http://eagle.strw.leidenuniv.nl/}
  and\\ \url{http://www.eaglesim.org}}.

For each galaxy in the database and at each redshift, we provide a wide range of
basic halo and galaxy properties such as stellar masses, gas masses, unextincted
magnitudes, angular momenta, star formation rates and $gri$ images, as well as
extensive information on metal abundances. Three additional tables give the
properties of galaxies measured in a series of apertures, more physically
motivated galaxy sizes and galaxy photometry. Using their merger trees, galaxies
can be tracked through time and their assembly history explored by analysing
their progenitors.

By making the halo and galaxy data public we hope that our simulations will be
helpful both for comparison with observational data, and as a tool for gaining
physical insight into the physics of galaxy formation. 

In Section~\ref{caveats} we presented some limitations of the simulations that
should be borne in mind when using the database. In particular, caution should
be exercised because of the finite resolution of the simulations. Over time we
intend to make additional data products available as the relevant papers are
accepted for publication. These will include, among other quantities, photometry
including dust extinction and information on the morphology of the galaxies. At
later stages, we may also release merger trees with higher time resolution, more
simulations models from \cite{Crain2015} as well as the raw particle data.

The \eagle database will hopefully be a powerful resource for the community to
explore the physics of galaxy formation, and to help interpret observational
data.

%|------------------|
%| Acknowledgements |
%|------------------|
\section*{Acknowledgements}
This work would have not be possible without Lydia Heck and Peter Draper's
technical support and expertise. We are grateful to all members of the Virgo
Consortium and the \eagle collaboration who have contributed to the development
of the codes and simulations used here, as well as to the people who helped with
the analysis. We thank Jaime Salcido for his help producing figure 3, Violeta
Gonzalez-Perez, Qi Guo and Claudia Lagos for useful comments on early drafts as
well as Chris Barber, Bart Clauwens and Sean McGee for testing earlier versions
of the \eagle database.  \\ This work was supported by the Science and
Technology Facilities Council (grant number ST/F001166/1); European Research
Council (grant numbers GA 267291 ``Cosmiway'' and GA 278594
``GasAroundGalaxies'') and by the Interuniversity Attraction Poles Programme
initiated by the Belgian Science Policy Office (AP P7/08 CHARM). RAC is a Royal
Society University Research Fellow.\\ This work used the DiRAC Data Centric
system at Durham University, operated by the Institute for Computational
Cosmology on behalf of the STFC DiRAC HPC Facility (www.dirac.ac.uk). This
equipment was funded by BIS National E-infrastructure capital grant
ST/K00042X/1, STFC capital grant ST/H008519/1, and STFC DiRAC Operations grant
ST/K003267/1 and Durham University. DiRAC is part of the National
E-Infrastructure.  We acknowledge PRACE for awarding us access to the Curie
machine based in France at TGCC, CEA, Bruy\`eres-le-Ch\^atel. \\ The web site
described in this paper was based on the one build for the \emph{Millennium
  Simulation} as part of the activities of the German Astrophysical Virtual
Observatory (GAVO).
