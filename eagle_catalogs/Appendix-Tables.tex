\onecolumn
\section{Description of all fields contained in the database}
\label{appendix_quantities}
%|----------------|
%| SubHalo Table. |
%|----------------|

% reset table counter
\setcounter{table}{0}
\renewcommand\thetable{\Alph{section}.\arabic{table}}
\begin{table*}[h]
\caption{Full listing of the content of the main galaxy properties table and
  description of the columns. These properties are contained in tables denoted
  {\bf [modelname]\_SubHalo}. The first five lines of the table give the
  indices used to navigate between the side tables and through the merger
  trees. Particle types are dark matter, gas, stars and black holes and
  collective properties such as Mass sum over all of these particles unless otherwise stated. }
\label{table:subhalo1}
\begin{center}
\footnotesize
\renewcommand{\arraystretch}{1.5}
\begin{tabular}{ >{\ttfamily}p{4cm}p{1.5cm}p{11cm}}
{\large \bf SubHalo} & & \\
\hline
\normalfont Field & Units & Description \\
\hline\hline
GalaxyID &
- &
Unique identifier of a galaxy. This identifier enables linking the {\bf
  SubHalo} table to the {\bf Aperture}, {\bf Magnitudes} and {\bf Sizes} tables.\\

LastProgID & - & Used for merger tree traversal, Section~\ref{subsection:merger_trees}. \\

TopLeafID & - & Used for merger tree traversal, Section~\ref{subsection:merger_trees}.\\

DescendantID & - & \GalaxyID of the descendant of this galaxy, Section~\ref{subsection:merger_trees}.\\

\hline

GroupID & - & Unique identifier of the FoF halo hosting this galaxy. This identifier enables linking 
the {\bf SubHalo} table to the {\bf FOF} table.\\

\hline

Redshift & - & Redshift at which these properties are computed. \\

Snapnum & - & Snapshot number at which these properties are computed. \\

\hline

GroupNumber & - & Integer identifier of the FoF halo hosting this galaxy. GroupNumber
is only unique to a given snapshot and hence cannot be used to identify the
same halo across multiple snapshots.\\

SubGroupNumber & - & Integer identifier of this galaxy within its FoF halo. SubGroupNumber is only unique to a given FoF halo in a given snapshot and hence cannot be used to identify the same galaxy across multiple outputs. The condition ``{\tt   SubGroupNumber}$~=~0$'' selects central galaxies.\\ 

\hline

Spurious & - & Value is $1$ if the galaxy is an artefact of
the \subfind algorithm and $0$ if the galaxy is a genuine object, see Section~\ref{caveats}.\\

\hline

Image\_face &  & \\
Image\_edge & - & \\
Image\_box &  & \multirow{-3}{11cm}{Weblink to the mock $gri$
image of the galaxy in the three different orientations (face-on, edge-on and along the simulation $z$-axis). When querying the database via the browser, the image appears in the column of the results table. } \\

\hline

BlackHoleMass &
M$_\odot$ &
Sum of all the black hole subgrid masses in this galaxy. See Eq.~(\ref{eq:m_bh}) for a description of the subgrid mass $\tilde m$ of a black hole, and Section~\ref{caveats} for cautions using black hole masses. \\

BlackHoleMassAccretionRate &
M$_\odot$~yr$^{-1}$ &
Total instantaneous accretion rate of all black holes, see Section~\ref{caveats} for cautions. \\

CentreOfMass\_x & &\\
CentreOfMass\_y & cMpc &\\
CentreOfMass\_z & & \multirow{-3}{11cm}{Co-moving position of the centre of mass, Eq.~(\ref{eq:CentreOfMass}).} \\

CentreOfPotential\_x & &\\
CentreOfPotential\_y & cMpc &\\
CentreOfPotential\_z & & \multirow{-3}{11cm}{Co-moving position of the minimum of the gravitational potential defined by the position of the most bound particle.} \\

\hline
 
\end{tabular}
\end{center}
\end{table*}

%|------------------|
%| SubHalo Table 2. |
%|------------------|
% table continued - so reset counter again
\setcounter{table}{0}
\renewcommand\thetable{\Alph{section}.\arabic{table}}
\begin{table*}
\caption{ -- continued}
\label{table:subhalo2}
\begin{center}
\footnotesize
\renewcommand{\arraystretch}{1.5}
\begin{tabular}{ >{\ttfamily}p{4cm}p{1.5cm}p{11cm}}
{\large \bf SubHalo} & & \\
\hline
\normalfont Field & Units & Description \\
\hline\hline


GasSpin\_x & & \\
GasSpin\_y & pkpc~km~s$^{-1}$ & \\
GasSpin\_z & & \multirow{-3}{11cm}{Spin per unit mass of all gas particles, ${\bf L}/M$, with ${\bf L}$ given by Eq.~(\ref{eq:spin}).}\\

HalfMassRad\_DM &
pkpc &
Physical radius enclosing half of the dark matter mass.\\

HalfMassRad\_Gas &
pkpc &
Physical radius enclosing half of the gas mass.\\

HalfMassRad\_Star &
pkpc &
Physical radius enclosing half of the stellar mass.\\

HalfMassRad\_BH &
pkpc &
Physical radius enclosing half of the black hole particle mass, $m$, as defined in Eq.~(\ref{eq:m_bh}). \\

HalfMassProjRad\_DM &
pkpc &
Projected physical radius enclosing half of the dark matter mass, averaged over three orthogonal projections.\\

HalfMassProjRad\_Gas &
pkpc &
Projected physical radius enclosing half of the gas mass, averaged over three orthogonal projections.\\

HalfMassProjRad\_Star &
pkpc &
Projected physical radius enclosing half of the stellar mass, averaged over three orthogonal projections.\\

HalfMassProjRad\_BH &
pkpc &
Projected physical radius enclosing half of the black hole particle mass, $m$ (Eq.~\ref{eq:m_bh}), averaged over three orthogonal projections.\\

InitialMassWeightedBirthZ &
z &
Mean redshift of formation of stars, weighted by birth mass $\tilde m$ (Eq.\ref{eq:m_star}). Calculated via $\sum_i \tilde {m}_{i} \tilde z_i / \sum_i \tilde {m}_{i}$ where $\tilde z_i$ is the redshift
the star particle $i$ was formed and $\tilde m_i$ its birth mass.\\

InitialMassWeightedStellarAge &
Gyr &
Mean age of stars, weighted by birth mass. Calculated via $\sum_i \tilde {m}_{i} (t - \tilde t_i) / \sum_i \tilde {m}_{i}$ where $t$ is cosmic time, and $\tilde t_i$ and $\tilde m_i$ formation time and birth mass of the star particle $i$, respectively.\\

KineticEnergy &
M$_{\odot}$~(km/s)$^{2}$ &
Total kinetic energy $E_K$, Eq.(\ref{eq:KineticEnergy}).\\

Mass &
M$_\odot$ &
Total current mass of all particles (i.e. $\sum_i m_i$ where $m_{i}$ is the mass of the particle).\\

MassType\_DM &
M$_\odot$ &
Total dark matter mass. \\

MassType\_Gas &
M$_\odot$ &
Total gas mass. \\

MassType\_Star &
M$_\odot$ &
Total stellar mass, $\sum_i m_i$, where $m_i$ is the stellar particle mass from Eq.~(\ref{eq:m_star}).\\

MassType\_BH &
M$_\odot$ &
Total black hole mass, $\sum_i m_i$, where $m_i$ is the black hole particle mass from Eq.~(\ref{eq:m_bh}).\\

RandomNumber &
- &
Random number uniform in the range $[0,1)$. \\

StarFormationRate &
M$_\odot$~yr$^{-1}$ &
Total star formation rate, $\sum_i \dot m_{\star, i}$, where $\dot m_{\star, i}$ is the star formation rate of gas particle $i$. \\

StellarInitialMass &
M$_\odot$ &
Sum of birth masses of all stars, $\sum_i \tilde{m}_{i}$, where $\tilde {m}_{i}$ is the birth mass of star particle $i$ from Eq.~(\ref{eq:m_star}).\\

StellarVelDisp &
km~s$^{-1}$ &
One dimensional velocity dispersion of stars, $((2\,\texttt{Star\_KineticEnergy})/(3\,\texttt{MassType\_Star}))^{1/2}$, where
${\tt Star\_KineticEnergy}$ is the kinetic energy of stars.\\

ThermalEnergy &
M$_\odot$~(km/s)$^{2}$ &
Total thermal energy $E_u$, Eq.~(\ref{eq:ThermalEnergy}).\\

TotalEnergy &
M$_\odot$~(km/s)$^{2}$ & 
Total energy $E_{\rm tot}$, Eq.~(\ref{eq:TotalEnergy}).\\

Velocity\_x & & \\
Velocity\_y & km~s$^{-1}$ & \\
Velocity\_z & & \multirow{-3}{11cm}{Peculiar velocity, Eq.~(\ref{eq:Velocity}).}\\

Vmax &
km~s$^{-1}$ & Maximum value of the circular velocity, $(G(M(<r)/r))^{1/2}$, where $M(<r)$ is the total mass enclosed in a sphere of physical radius $r$.\\

VmaxRadius &
pkpc &
Physical radius where the circular velocity equals \texttt{Vmax}.\\

\hline

\end{tabular}
\end{center}
\end{table*}

%|------------------|
%| SubHalo Table 3. |
%|------------------|
% table continued - reset counter
\setcounter{table}{0}
\renewcommand\thetable{\Alph{section}.\arabic{table}}
\begin{table*}
\caption{ -- continued. Columns in this table exist for each of three different components:
star-forming gas ({\tt SF}), non-star-forming gas ({\tt NSF}) and
stars ({\tt Stars}). As these properties are repeated for each of these components,
we only describe them once. In the database each property will be preceded with
either {\tt [SF/NSF/Stars]\_} before its name. For instance, the metallicity
field will exist in three variants: {\tt SF\_Metallicity}, {\tt NSF\_Metallicity}
and {\tt Stars\_Metallicity} for the metallicity of the star-forming gas, of the
non star-forming gas and of the stars, respectively. Any sum used to describe a property is the sum of all particles for that component only. }
\label{table:subhalo3}
\begin{center}
\footnotesize
\renewcommand{\arraystretch}{1.5}
\begin{tabular}{ >{\ttfamily}p{4cm}p{1.5cm}p{11cm}}
{\large \bf SubHalo} & & \\
\hline
\normalfont Field & Units & Description \\
\hline\hline

Hydrogen & - & \\
Helium & - & \\
Carbon & - & \\
Nitrogen & - & \\
Oxygen & - & \\
Neon & - & \\
Magnesium & - & \\
Silicon & - & \\
Sulphur & - & \\
Calcium & - & \\
Iron & - & \multirow{-11}{11cm}{Total mass in this element divided by the total mass (both for a given component). These are therefore absolute abundances which do not depend on the solar abundance.} \\

IronFromSNIa &
- &
Total mass in {\tt Iron} contributed by ejecta from Type Ia supernovae, divided by the total mass.\\

KineticEnergy &
M$_{\odot}$~(km/s)$^{2}$ &
Total kinetic energy $E_K$, Eq.~(\ref{eq:KineticEnergy}).\\

Mass &
M$_{\odot}$ &
Total mass, $\sum_i m_i$, where $m_i$ is the particle mass.\\

MassFromAGB &
M$_{\odot}$ &
Total mass contributed by ejecta of AGB stars.\\

MassFromSNII &
M$_{\odot}$ & 
Total mass contributed by ejecta from massive stars and type II supernovae.\\

MassFromSNIa &
M$_{\odot}$ & 
Total mass contributed by ejecta from type Ia SN supernovae. \\

MassWeightedEntropy &
km~s$^{2}$~$(\frac{10^{10} \Msol}{\rm Mpc})^\frac{2}{3}$ &
~~~~~~Mass-weighted pseudo entropy of all particles, $\sum_i m_i S_i / \sum_i m_i$, where $S_{i}$ is the pseudo-entropy (Eq.~\ref{eq:S}) and $m_{i}$ is the mass of the particle $i$. (Entry present for gas components only.)\\

MassWeightedTemperature &
K &
Mass-weighted temperature, $\sum_i m_i \mathrm{T}_{i} / \sum_i m_{i}$, where $\mathrm{T}_{i}$ is the temperature and $m_{i}$ is the mass of the particle $i$. (Entry present for gas components only.)\\

Metallicity &
- & 
Metal mass fraction, $\sum_i m_i Z_i / \sum_i m_i$, where $Z_{i}$ is the metallicity and $m_{i}$ is the mass of the particle $i$. \\

MetalsFromAGB &
M$_\odot$ & 
Total metal mass contributed by ejecta from AGB stars. \\

MetalsFromSNII &
M$_\odot$ & 
Total metal mass contributed by ejecta from massive stars and SN Type II supernovae.\\

MetalsFromSNIa &
M$_\odot$ & 
Total metal mass contributed by ejecta from Type Ia supernovae.\\

Spin\_x & & \\
Spin\_y & pkpc~km~s$^{-1}$ &  \\
Spin\_z & & \multirow{-3}{11cm}{Spin per unit mass, ${\bf L}/M$, from Eq.~(\ref{eq:spin}).}\\

ThermalEnergy & 
M$_\odot$~(km/s)$^{2}$ &
Total thermal energy $E_u$, Eq.~(\ref{eq:ThermalEnergy}). (Entry present for gas components only.)\\

TotalEnergy & 
M$_\odot$~(km/s)$^{2}$ &
Total energy $E_{\rm tot}$, Eq.~(\ref{eq:TotalEnergy}). The potential energy contribution does include the other components as well, $E_{\Phi}={1\over 2}\sum_i m_i{\hat\Phi_i\over a}$.\\

\hline 

\end{tabular}
\end{center}
\end{table*}

%|------------|
%| FOF Table. |
%|------------|
\begin{table*}
\caption{Full listing of the content of the halo table and description of
  the columns. These properties are contained in tables denoted {\bf
    [modelname]\_FOF}. This table can be linked to the {\bf
    [modelname]\_SubHalo} table using the unique {\tt GroupID}
    identifier.}
\label{table:fof}
\begin{center}
\footnotesize
\renewcommand{\arraystretch}{1.5}
\begin{tabular}{ >{\ttfamily}p{4cm}p{1.5cm}p{11cm}}
{\large \bf FOF} & & \\
\hline
\normalfont Field & Units & Description \\
\hline\hline

GroupID &
- &
Unique identifier of a halo (i.e. a FoF group). This identifier enables linking a halo
to all its galaxies and their properties in the {\bf SubHalo} table.\\

\hline

Redshift & - & Redshift at which these properties are computed. \\

SnapNum & - & Snapshot number containing that halo. \\

\hline

GroupCentreOfPotential\_x &
 & \\
GroupCentreOfPotential\_y &
cMpc & \\
GroupCentreOfPotential\_z & & \multirow{-3}*{Co-moving position of the minimum of the gravitational potential of the halo.}\\

GroupMass &
M$_{\odot}$ &
Total Friends-of-Friends mass of this halo. \\

Group\_M\_Crit200  \\

Group\_M\_Crit500  &  M$_{\odot}$ \\

Group\_M\_Crit2500 & & \multirow{-3}{11cm}{Total mass within the corresponding {\tt Group\_R\_Critxxx} radius, where xxx=200, 500 or 2500, respectively} \\


Group\_M\_Mean200 \\ 
Group\_M\_Mean500 & M$_{\odot}$ \\
Group\_M\_Mean2500 & & \multirow{-3}{11cm}{Total mass within the corresponding {\tt Group\_R\_Meanxxx} radius, where xxx=200, 500 or 2500, respectively} \\


Group\_M\_TopHat200 &
M$_{\odot}$ &
Total mass within radius {\tt Group\_R\_Tophat200}. \\

Group\_R\_Crit200 \\
Group\_R\_Crit500 & pkpc \\
Group\_R\_Crit2500 & & \multirow{-3}{11cm}{Physical radius within which the mean density is xxx times the \emph{critical} density of the Universe, where xxx=200, 500 or 2500, respectively}.\\

Group\_R\_Mean200\\
Group\_R\_Mean500 & pkpc \\
Group\_R\_Mean2500 & & \multirow{-3}{11cm}{Physical radius within which the mean density is xxx times the \emph{mean} density of the Universe, where xxx=200, 500 or 2500, respectively}.\\

Group\_R\_TopHat200 &
pkpc &
Physical radius within which the mean density is $18\pi^{2} + 82
(\Omega_{m}(z)-1)-39 (\Omega_{m}(z)-1)^{2}$ times the critical density of the Universe. This is based on 
the spherical top-hat collapse model of \citep{Bryan1998}.\\

NumOfSubhalos &
- &
Number of subhaloes (galaxies) identified as belonging to this halo. \\

RandomNumber &
- &
Random number uniform in the range $[0,1)$. \\
\hline
\end{tabular}
\end{center}
\end{table*}

%|--------------|
%| Sizes Table. |
%|--------------|
\begin{table*}
\caption{Full listing of the content of the galaxy sizes table and description of
  the columns. These properties are contained in tables denoted {\bf
    [modelname]\_Sizes}. This table contains half-mass sizes of
    the {\em stellar} component of galaxies using spherical apertures (Furlong et al. (\textit{in prep.}). The \GalaxyID column can be
    used to join this table to the corresponding {\bf
    [modelname]\_SubHalo} table. Only galaxies with total stellar mass $M_*>10^8\Msol$
    have entries in this table. }
\label{table:sizes}
\begin{center}
\footnotesize
\renewcommand{\arraystretch}{1.5}
\begin{tabular}{ >{\ttfamily}p{4cm}p{1.5cm}p{11cm}}
{\large \bf Sizes} & & \\
\hline
\normalfont Field & Units & Description \\
\hline\hline

GalaxyID &
- &
Unique identifier of a galaxy as per {\bf SubHalo} table. \\

\hline

R\_halfmass30 &
pkpc &
Half mass radius of stellar component within a spherical (3D) 30~pkpc aperture.\\

R\_halfmass100 &
pkpc &
Half mass radius of stellar component within a spherical (3D) 100 pkpc aperture.\\

R\_halfmass30\_projected &
pkpc &
Projected half mass radius of stellar component within a circular (2D) 30~pkpc aperture
(averaged over three orthogonal projections).\\

R\_halfmass100\_projected & pkpc & Projected half mass radius of stellar component within a circular
(2D) 100~pkpc aperture (averaged over three orthogonal
projections).\\
\hline

\end{tabular}
\end{center}
\end{table*}






%|-----------------|
%| Aperture Table. |
%|-----------------|
\begin{table*}
\caption{Full listing of the content of the aperture table and description of
  the columns. These properties are contained in tables denoted {\bf
    [modelname]\_Aperture}. This table contains measurements within spherical
  apertures centred on the minimum of the gravitational potential of a given galaxy. Each row
  represents a set of measurements for a single galaxy using a single aperture
  size in physical kpc. The \GalaxyID column can be used to join this table to
  the corresponding {\bf [modelname]\_SubHalo} table.}
\label{table:aperture}

\begin{center}
\footnotesize

\renewcommand{\arraystretch}{1.5}
\begin{tabular}{ >{\ttfamily}p{4cm}p{1.5cm}p{11cm}}
{\large \bf Aperture} & & \\
\hline
\normalfont Field & Units & Description \\
\hline\hline

GalaxyID &
- &
Unique identifier of a galaxy as per {\bf SubHalo} table.\\

\hline

ApertureSize &
pkpc &	
Spherical (3D) aperture radius used for this measurement. Quantities are measure in a sphere centred at the centre of the potential, {\em i.e.} at the location of the most bound particle. Available aperture sizes are: 1, 3, 5, 10, 20, 30, 40, 50, 70 and 100~pkpc. \\

VelDisp &
km~s$^{-1}$ &
One dimensional velocity dispersion of stars, $((2\,\textrm{KineticEnergy\_Star})/(3\,\textrm{Mass\_Star}))^{1/2}$, where ${\rm KineticEnergy\_Star}$ is the kinetic energy of stars, and the sum is over stars within the aperture.\\

SFR &
M$_{\odot}$~yr$^{-1}$ &
Star formation rate within the aperture. \\

Mass\_BH &
M$_{\odot}$ &	
Total particle mass, $\sum_i m_i$ (Eq.~\ref{eq:m_bh}), of all black holes within the aperture. \\

Mass\_DM &
M$_{\odot}$ &	
Total dark matter mass within the aperture. \\

Mass\_Gas &
M$_{\odot}$ &	
Total gas mass within the aperture. \\

Mass\_Star &
M$_{\odot}$ &	
Total stellar mass, $\sum_i m_i$ (Eq.~\ref{eq:m_star}), within the aperture. \\

\hline
\end{tabular}
\end{center}
\end{table*}








%|-----------------|
%| Magnitude Table.|
%|-----------------|
\begin{table*}
\caption{Full listing of the content of the magnitudes table and description of
  the columns. These properties are contained in tables denoted {\bf
    [modelname]\_Magnitudes}. This table contains absolute rest-frame magnitudes without dust
    attenuation for all galaxies with $M_*>10^{8.5}\Msol$ contained in the {\bf
    SubHalo} table. This table can be joined to the {\bf SubHalo} table using
    the \GalaxyID field. The magnitudes in the different
    SDSS \citep{SDSSfilters} and UKIRT \citep{UKIRTfilters} filters have been
    computed in 30~pkpc spherical apertures following the procedure described
    by \citet{Trayford2015}.}
\label{table:magnitudes}
\begin{center}
\footnotesize
\renewcommand{\arraystretch}{1.5}
\begin{tabular}{ >{\ttfamily}p{4cm}p{1.5cm}p{11cm}}
{\large \bf Magnitudes} & & \\
\hline
\normalfont Field & Units & Description \\
\hline\hline

GalaxyID &
- &
Unique identifier of a galaxy as per {\bf SubHalo} table.\\

\hline
u\_nodust & mag & Rest-frame absolute magnitude (AB) in the $u$ band without dust
attenuation. \\
g\_nodust & mag & Rest-frame absolute magnitude (AB) in the $g$ band without dust
attenuation. \\
r\_nodust & mag & Rest-frame absolute magnitude (AB) in the $r$ band without dust
attenuation. \\
i\_nodust & mag & Rest-frame absolute magnitude (AB) in the $i$ band without dust
attenuation. \\ 
z\_nodust & mag & Rest-frame absolute magnitude (AB) in the $z$ band without dust
attenuation. \\ 
Y\_nodust & mag & Rest-frame absolute magnitude (AB) in the $Y$ band without dust
attenuation. \\ 
J\_nodust & mag & Rest-frame absolute magnitude (AB) in the $J$ band without dust
attenuation. \\ 
H\_nodust & mag & Rest-frame absolute magnitude (AB) in the $H$ band without dust
attenuation. \\ 
K\_nodust & mag & Rest-frame absolute magnitude (AB) in the $K$ band without dust
attenuation. \\ 
\hline

\end{tabular}
\end{center}
\end{table*}


\twocolumn
