%|-----------------------------|
%| Recommendations and Caveats |
%|-----------------------------|
\section{Recommendations, caveats and credits}
\label{caveats}

\subsection{Caveats regarding the usage of the data}

In this section we list a series of recommendations and known limitations that the
authors have uncovered while working on the analysis of the simulation and the
preparation of the database. These points should be taken into consideration to
exploit the simulation outputs fully and to avoid mistakes in the interpretation
of the results.

\paragraph{\bf Finite resolution}
When using the galaxy catalogues, it should be remembered that the properties of
low-mass galaxies should be treated with caution. Large numbers of particles are
required to adequately sample the formation history of a galaxy. In general, we
find that many galaxy properties are unreliable below a stellar mass of $10^{9}
\Msol$ for the intermediate resolution simulations \citep{Schaye2015}. For any
given quantity, these effects can be assessed by comparing the Ref-L0025N0376
simulation with the higher-resolution Recal-L0025N0752 and Ref-L0025N0752
simulations.

\paragraph{\bf Finite volume}
Although the main simulation is one of the largest of its kind, its volume is
still only $10^{-3}~\rm{cGpc}^3$, a volume much smaller than the volumes
typically probed by surveys of the extragalactic Universe. This implies that
rare objects are unlikely to be found in the simulation volume. Moreover, due to
missing large-scale modes, the number density of rare objects will typically be
underestimated. Only a handful of haloes with mass $M_{200}$
(\texttt{Group\_M\_Crit200} in the {\bf FOF} table) above $10^{14}\Msol$ are
present in the main simulation, limiting the analysis of cluster-like
objects. The convergence with box size can be assessed by comparing the main
simulation to the smaller volumes that use the same resolution.

\paragraph{\bf Aperture masses and SFRs} 
The stripping of satellite galaxies as they orbit within a halo generates a
significant mass loss at large radii. The resulting diffuse light (and any
diffuse star formation) is extremely difficult to observe and is not commonly
included in observational galaxy catalogues.  Furthermore, the total galaxy
stellar masses and star formation rates can depend strongly on the precise
assignment of particles to the main subhalo within each FOF group by the
\subfind algorithm, which can lead to spurious total mass evolution.  For these
reasons, studies published by the EAGLE team use aperture masses and star
formation rates, typically in an aperture of $30~\rm{pkpc}$. As discussed by
\cite{Schaye2015}, this corresponds roughly to an $R_{80}$ Petrosian aperture
and is hence particularly well-suited to comparison with observations. We
recommend the use of aperture values when available.

\paragraph{\bf Self-bound star clusters and black holes}
As discussed by \cite{Schaye2015}, small dense stellar regions within galaxies
may occasionally be identified by \subfind as distinct subhaloes and hence
'galaxies'. These appear in the catalogue as rather unusual objects with little
stellar mass but anomalously high metallicity or black hole mass. These
``spurious'' galaxies are flagged in the database in the column {\tt Spurious}
(see the table in \ref{appendix_quantities}). Such objects should not be
considered as genuine galaxies and should be discarded from samples of simulated
galaxies.

\paragraph{\bf Black hole masses and accretion rates}
The black hole masses given in the main table (table {\bf SubHalo}, column {\tt
  BlackHoleMass}) do not directly correspond to the mass of the central
supermassive black hole of a galaxy, but to a summed value of all black holes
assigned to that subhalo. For cases where {\tt
  BlackHoleMass} $> 10^6\Msol$ this closely approximates the mass of
the most massive black hole. {\tt MassType\_BH} refers to the sum of the black
hole particle masses (see \ref{appendix_cosmo} for details of particle and
subgrid masses) and therefore should not be used for a galaxy's black hole
mass. Similarly, due to the coarse time sampling of the outputs, the high
temporal variability of the black hole accretion rates cannot be captured in the
database outputs and as such the quantity {\tt BlackHoleMassAccretionRate} should be
treated with great care.

\paragraph{\bf Stellar velocity dispersion and morphology}
The field {\tt StellarVelocityDispersion} stored in the {\bf SubHalo} table
is a measure of the kinetic energy of the stars, $\sigma = \sqrt{2 E_{\rm{K}} /
  3 M}$, and not a measure of the amount of stellar kinetic energy in dispersion
as opposed to rotation. In particular, it cannot be used to distinguish
rotationally supported galaxies (spirals) from dispersion supported galaxies
(ellipticals).

\paragraph{\bf Galaxy images and magnitude tables}
The images provided in the database are generated using only the particles
within a particular subhalo, in order to correspond with an entry in the
database tables. As a result satellites or merging partners may not be visible
in the images. While the images are observed as if redshifted to $z=0.1$ to
approximate typical SDSS colours, the magnitude tables are measured in the
rest-frame. The inclusion of different population synthesis models, dust
absorption and the relative scaling of images also implies that images are not
reducible to magnitude table entries. 

\paragraph{\bf This simulation is not the real Universe} 
The papers presenting \eagle have shown that the simulation broadly reproduces a
wide set of observational properties of galaxies and the intergalactic
medium. When using the database it should nevertheless be remembered that there
are known discrepancies between the simulation results and observational
data. In particular, we highlight the following points:


\begin{itemize}
\item Although the $z=0.1$ stellar mass function was used in the calibration of
  the simulation, the stellar mass density is approximately $20\%$ lower than
  inferred from observations \citep{Schaye2015,Furlong2015}. This missing mass
  can be related to the slight undershoot of the ``knee'' of the simulated
  galaxy stellar mass function.

\item The evolution of specific star formation rates broadly follows the trends
  seen in observational data, but with a normalisation lower by, depending on
  redshift, 0.3 - 0.5 dex \citep{Schaye2015, Furlong2015}. Note, however, that
  the \eagle galaxies are in good agreement with the recent recalibration of
  star formation indicators by \cite{Chang2015} \citep[see Fig. 5 of ][]{Schaller2015b}.

\item The present-day stellar mass -- metallicity relation in the
  intermediate-resolution Ref- model is flatter than the one inferred from
  observational data \citep{Schaye2015}.  Note, however, that the relation
  becomes steeper in the higher-resolution Recal-L0025N0752 simulation, in
  agreement with the observations.

\item The transition from active to passive galaxies occurs at too high a
  stellar mass at $z=0$ \citep{Schaye2015, Trayford2015}.

\end{itemize}

This list of flaws is certainly not exhaustive. Future papers will
undoubtedly uncover further deficiencies.

\subsection{Acknowledgement of usage}
To recognise the effort of the individuals involved in the design and execution
of these simulations, in their post-processing and in the construction of the
database, we kindly request the following:
\begin{itemize}

\item Publications making use of the \eagle data extracted from the public
  database are kindly requested to cite the original papers introducing the project
  \citep{Schaye2015,Crain2015} as well as this paper (McAlpine et al., 2015).

\item Publications making use of the database should add the following line in
  their acknowledgement section: ``\textit{We acknowledge the Virgo Consortium
    for making their simulation data available. The \eagle simulations were
    performed using the DiRAC-2 facility at Durham, managed by the ICC, and the
    PRACE facility Curie based in France at TGCC, CEA, Bruy\`eres-le-Ch\^atel.}''.

\item Furthermore, publications referring to specific aspects of the subgrid
  models, hydrodynamics solver, or post-processing steps (such as the
  construction of images or photometric quantities, and the construction of
  merger trees), are kindly requested to not only cite the above papers, but
  also the original papers describing these aspects. The appropriate references
  can be found in section 2 of this paper and in \cite{Schaye2015}.

\end{itemize}

