%|----------|
%| Abstract |
%|----------|
\begin{abstract}
We present the public data release of halo and galaxy catalogues extracted from
the \eagle suite of cosmological hydrodynamical simulations of galaxy
formation. These simulations were performed with an enhanced version of the
\gadget code that includes a modified hydrodynamics solver, time-step limiter
and subgrid treatments of baryonic physics, such as stellar mass loss,
element-by-element radiative cooling, star formation and feedback from star
formation and black hole accretion. The simulation suite includes runs performed
in volumes ranging from 25 to 100 comoving megaparsecs per side, with numerical
resolution chosen to marginally resolve the Jeans mass of the gas at the star
formation threshold. The free parameters of the subgrid models for feedback are
calibrated to the redshift $z=0$ galaxy stellar mass function, galaxy sizes and
black hole mass - stellar mass relation. The simulations have been shown to
match a wide range of observations for present-day and higher-redshift galaxies.
The raw particle data have been used to link galaxies across redshifts by
creating merger trees. The indexing of the tree produces a simple way to connect
a galaxy at one redshift to its progenitors at higher redshift and to identify
its descendants at lower redshift. In this paper we present a relational
database which we are making available for general use. A large number of
properties of haloes and galaxies and their merger trees are stored in the
database, including stellar masses, star formation rates, metallicities,
photometric measurements and mock $gri$ images. Complex queries can be created
to explore the evolution of more than $10^5$ galaxies, examples of which are
provided in appendix. The relatively good and broad agreement of the simulations
with a wide range of observational datasets makes the database an ideal resource
for the analysis of model galaxies through time, and for connecting and
interpreting observational datasets.
\end{abstract}

\begin{keyword}
cosmology: theory - galaxies: formation - galaxies: evolution - method: numerical
\end{keyword}

\end{frontmatter}

%|--------------|
%| Introduction |
%|--------------|
\section{Introduction}

Galaxy formation is a complex, non-linear process that involves a wide range of
physical and astrophysical phenomena, from the evolution of dark matter
clustering to intricate feedback effects coupling gas cooling and outflows to
star and black hole formation. Theoretical studies of galaxy formation thus
require rigorous detailed modelling to link together these phenomena over a very
wide range of scales. Two techniques have been developed for this purpose:
semianalytic modelling \citep{White1991} and hydrodynamical simulations
\citep{Carlberg1990,Katz1992}. Both techniques have been extensively developed
over the past 25 years \citep[e.g.][for semi-analytic
  models]{Porter2014,Henriques2015,Lacey2015} and \citep[e.g.][for hydrodynamical
  simulations]{Oppenheimer2010,Puchwein2013,
  Dubois2014,Okamoto2014,Vogelsberger2014,Khandai2015}.

Recently, the Virgo\footnote{\url{http://virgo.dur.ac.uk/}} Consortium's
``Evolution and Assembly of GaLaxies and their Environments'' simulation suite
\citep[\eagle,][]{Schaye2015,Crain2015} has been able to reproduce key
observational datasets, such as the present-day stellar mass function of
galaxies, the correlation of black hole mass and stellar mass and the dependence
of galaxy sizes on stellar mass, with unprecedented fidelity. As well as
reproducing these observations, which were used during the calibration of the
simulation parameters, the simulation outputs match many other properties of the
observed galaxy population and the intergalactic medium both at the present day
and at earlier epochs, as we briefly discuss below. These simulations therefore
provide a powerful resource for understanding the formation of galaxies and for
linking and interpreting observational datasets.

The aim of this paper is to introduce and make available a relational database
that can be queried using the Structured Query Language (\sql) to explore and
exploit the halo and galaxy catalogues of the main \eagle simulations. Columns
containing integrated quantities describing the galaxies, such as stellar mass,
star formation rates, metallicities and luminosities, are provided for more than
$10^5$ simulated galaxies and these can be individually followed through their
evolution across cosmic time. This database is available at the address
\dbAddress.

The simulations follow the gravitational hydrodynamical equations, tracking the
evolution of baryons and dark matter. The initial conditions reflect the small
density fluctuations observed in the cosmic microwave background (CMB). By
tracking the movement of baryon and dark matter particles, the simulations
calculate how these fluctuations are amplified by gravity, and how pressure and
radiative cooling of baryons separate these two matter components of the
universe. The simulations include subgrid formulations to account for processes
that cannot be directly resolved in the calculation and that describe how stars
and black holes form and impact the matter distribution around them. \eagle
improves on previous hydrodynamical simulations of representative volumes,
through the use of physically motivated subgrid source and sink terms as well as
through the adoption of a clear strategy for the calibration of uncertain
subgrid parameters \citep{Crain2015} and by producing a galaxy population that
reproduces many of the characteristics of the observed population over a wide
range of redshifts.

~\\

The usability of the simulation data products is greatly enhanced when presented
in a relational database, making it simple and quick to select galaxy samples
based on multiple galaxy properties, to connect them to their halos and to
follow their evolution over cosmic time \citep{Lemson2006a}.  Such databases
were originally designed to host results from large surveys \citep[e.g. the SDSS
  SkyServer][]{Szalay2000} and later the halo catalogues from dark matter
simulations and galaxy catalogues from semi-analytic models \citep[applied to
  the \emph{Millennium Simulation}, see][]{Lemson2006b}. They have since been
expanded to include the wider range of data available from hydrodynamical
simulations \citep[e.g.][]{Dolag2009,Khandai2015,Nelson2015}. The database
allows multiple indexing of the data that significantly enhances access speed
and allows the selection of smaller data subsets that can be quickly analysed
using simple scripting languages. This approach avoids the need for the user to
copy the raw simulation data or even just the full galaxy catalogues, reducing
data transfer volumes to a manageable level. The galaxy properties stored in the
database can be compared to observations or to other models, whilst the physics
of galaxy formation can be explored by tracking an individual galaxy's behaviour
and environment through cosmic time.

This paper is intended as a reference guide for accessing the publicly available
\eagle database, and is laid out as follows. Section \ref{eagle_sim_suite}
presents a brief overview of the \eagle simulation suite, including the list of
simulations available in the database and the values of the subgrid parameters
that vary, as well as an overview of the construction of the merger trees and
database tables.  A short tutorial describing how to access the data is
presented in Section \ref{database_usage}. We give some words of caution and
some remarks on the simulations in Section~\ref{caveats} and conclude in
Section~\ref{conclusion}. Some additional examples combining the {\sc python}
and \sql languages to access the data are given in \ref{example_scripts} whilst
the full list of galaxy and halo properties available in this data release is
given in \ref{appendix_quantities} together with a list of output redshifts in
\ref{appendix_snapshot} and detailed equations given in \ref{appendix_cosmo}.
Throughout this paper we quote magnitudes in the AB system and use
\squotes{h-free} units unless stated otherwise.
