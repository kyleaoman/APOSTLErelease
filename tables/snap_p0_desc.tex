\begin{tabular}{>{\ttfamily}p{4cm}p{1.5cm}p{11cm}}
\multicolumn{3}{l}{\large \bf PartType0: Gas} \\
\hline
Field & Equation/Par & Description \\ \hline\hline

AExpMaximumTemperature &
\S\ref{SecThermo} &
Expansion factor when particle had maximum temperature. \\

Coordinates &
Eq.~(\ref{EqCoords}) &
\coordinates \\

Density &
Eq.~(\ref{EqRho}) &
Co-moving density. \\

ElementAbundance &
\S\ref{SecAbundances} &
Mass of [Carbon, Helium, Hydrogen, Iron, Magnesium, Neon, Nitrogen, Oxygen, Silicon] divided by particle mass. \\

Entropy &
\S\ref{SecThermo} &
Particle entropy. \\

GroupNumber & \S\ref{SecFOF} & \groupnumber \\

HostHalo\_TVir\_Mass &
- &
Estimate of host FoF group's virial temperature, calculated from the local velocity dispersion. \\

InternalEnergy &
Eq.~(\ref{EqU}) &
Thermal energy per unit mass. \\

IronMassFracFromSNIa &
\S\ref{SecAbundances} &
Mass of Iron from SNIa divided by particle mass. \\

Masses &
\S\ref{SecParticles} &
This variable was called `\texttt{Mass}' in \eagle. Particle mass.\\

MaximumTemperature &
S\ref{SecThermo} &
Maximum temperature ever reached by particle. \\

MetalMassFracFromAGB &
\S\ref{SecAbundances} &
Mass of metals received from AGB divided by particle mass. \\

MetalMassFracFromSNII &
\S\ref{SecAbundances} &
Mass of metals received from SNII divided by particle mass. \\

MetalMassFracFromSNIa &
\S\ref{SecAbundances} &
Mass of metals received from SNIa divided by particle mass. \\

Metallicity &
\S\ref{SecAbundances} & Mass of elements heavier than Helium, including
those not tracked individually, divided by particle mass\\

ParticleIDs &
\S\ref{SecParticles} & Unique particle identifier. Index encodes the particles position in the
initial conditions (see Appendix of \cite{2015MNRAS.446..521S} for details). \\

SfFlag &
\S\ref{SecSFR} & This variable was called `\texttt{OnEquationOfState}' in \eagle. 0 if particle has never been star-forming, +ve if currently star-forming,
-ve if not currently star-forming. Value indicates scale factor at which it
obtained its current state. Note this does not ensure the gas particle is on
the equation of state for +ve values, as gas particles can yield non-zero star
formation rates  up to 0.5~dex above the equation of state. \\

SmoothedElementAbundance &
\S\ref{SecAbundances}  & SPH kernel weighted ElementAbundance (see Section 2.2 of \cite{2005MNRAS.364.1105S} for
the description of kernel weighted properties in SPH). \\

SmoothedIronMassFracFromSNIa &
\S\ref{SecAbundances}  & SPH kernel weighted IronMassFracFromSNIa. \\

SmoothedMetallicity &
\S\ref{SecAbundances}  &
SPH kernel weighted Metallicity. \\

SmoothingLength &
\S\ref{SecParticles} &
Co-moving SPH smoothing kernel. \\

StarFormationRate &
Eq.~(\ref{EqSFLaw}) &
Instantaneous star formation rate. \\

SubGroupNumber &
\S\ref{SecFOF} & \subgroupnumber \\

Temperature &
\S\ref{SecThermo} &
Temperature \\

TotalMassFromAGB &
\S\ref{SecAbundances} &
Total mass received from AGB. \\

TotalMassFromSNII &
\S\ref{SecAbundances} &
Total mass received from SNII. \\

TotalMassFromSNIa &
\S\ref{SecAbundances} &
Total mass received from SNIa. \\

Velocities &
S\ref{SecParticles} &
\velocity \\

\hline
\end{tabular}
\end{center}
\end{table}

\begin{table}
\label{TabDM}
\caption{Description and equation, where applicable, for each property of dark matter
(PartType1) particles.}
\begin{center}
\footnotesize
\renewcommand{\arraystretch}{1.5}
\begin{tabular}{>{\ttfamily}p{4cm}p{1.5cm}p{11cm}}
\multicolumn{3}{l}{\large \bf PartType1: Dark Matter} \\
\hline
Field & Equation/Par & Description \\ \hline\hline

Coordinates &
\S\ref{SecParticles} &
\coordinates \\

GroupNumber &
\S\ref{SecFOF} &
\groupnumber \\

ParticleIDs &
\S\ref{SecFOF} & Unique particle identifier. Index encodes the particles position in the
initial conditions (see \cite{2015MNRAS.446..521S} for details). \\

SubGroupNumber &
\S\ref{SecFOF} &
\subgroupnumber \\

Velocities &
\S\ref{SecParticles} &
\velocity \\

\hline
\end{tabular}
